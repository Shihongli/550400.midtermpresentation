\documentclass[compress,handout,10pt]{beamer}

\newlength{\wideitemsep}
\setlength{\wideitemsep}{\itemsep}
\addtolength{\wideitemsep}{100pt}
\let\olditem\item
\renewcommand{\item}{\setlength{\itemsep}{0.5\baselineskip}\olditem}

\usetheme{Singapore}
\usecolortheme{rose}
\usefonttheme[onlymath]{serif}

\usepackage{float}
\floatstyle{boxed}
\usepackage{epsfig}
\usepackage{colortbl}
\usepackage{mathpazo}
\usepackage{graphicx}
\usepackage{movie15}
\usepackage{bm}
\usepackage{verbatim}
\usepackage{comment}
\usepackage{caption}
\usepackage{subcaption}
\captionsetup[subfigure]{labelformat=empty}
\captionsetup[figure]{labelformat=empty}

\newcommand{\mygreen}{\color{green!50!black}}
\newcommand{\myblue}{\color{blue}}
\newcommand{\myred}{\color{red}}
\newcommand{\mycolor}{\color{red}{c}\color{blue}{o}\color{green}{l}\color{orange}{o}\color{cyan}{r}}
\newcommand{\mysize}{\scriptsize{s}\small{i}\normalsize{z}\Large{e}}
\newcommand{\myshape}{\textcircled{s}\textit{h}\texttt{a}\textsf{p}\textsc{e}}

\title{{\color{black} \LARGE Statistical Arbitrage in Futures Market\newline} }

\subtitle{{\color{black} Midterm Presentation\large } }

\author{ 
%    \vspace{5pt}
    {\bf{Participant:}} \\ 
Shihong Li\\ 
    \vspace{5pt}
    {\bf{Supervisor:}} \\ 
Nam Lee\\ 
    \vspace{5pt}
} 
\institute{
{\bf{Sponsor:}}\\
Greenwoods Asset Management Ltd.\\
\vspace{5pt}
}
\xdefinecolor{titlecolor}{rgb}{.855,.647,.125}
\setbeamercolor{frametitle}{fg=titlecolor}
\setbeamerfont{frametitle}{series=\bfseries}
\setbeamercolor{normal text in math text}{parent=math text}






\date{\mygreen Prepared Date: \today} 

\begin{document}

\begin{frame}[plain]
    \titlepage
\end{frame}

\begin{frame}
    \frametitle{CONTENT}
    \tableofcontents
\end{frame}

\section{Introduction}

\begin{frame}
    \frametitle{1. Terminology}
{\bf{Arbitrage}} \\The possibility of a risk-free profit at zero cost. \\
\vspace{7pt}
{\bf{Statistical Arbitrage}}
             \begin{enumerate}
                 \item An investment process based on mathematical models
                 \item Aiming at making profits
                 \item Building up long and short positions for assets
                 \item Taking advantage of asset prices` deviation from theoretical values
             \end{enumerate}
\end{frame}

\begin{frame}
    \frametitle{2. Background Information}
{\bf{Prerequisite:}}\\
\vspace{3pt}
The securities market in which short selling exists.\\
\vspace{7pt}
{\bf{Financial Market in China:}}\\
    \begin{enumerate}
        \item Before: 
            \begin{enumerate}
                \item Absent of short selling mechanism
                \item Absent of stock index futures
            \end{enumerate}
        \item Now:
            \begin{enumerate}
                \item Launched stock index futures on April 16 2010
                \item Inprovement in startup of securities margin trading
            \end{enumerate}
    \end{enumerate}
\end{frame}

\begin{frame}
    \frametitle{3. Sponsor`s Background}
{\bf{Greenwoods Asset Management Ltd.}}\\
\vspace{7pt}
 An investment management company specializing in managing investments into mainland China companies. Greenwoods currently manage funds investing in Greater China equities for global investors and A-share trusts for qualified Chinese domestic investors.
\end{frame}

\begin{frame}
    \frametitle{4. Deliverables}
{\bf{From team to sponsor:}}\\
\vspace{3pt}
    \begin{enumerate}
 \item We are going to present an algorithm along with a model in the end of this
project.
\item The spread of featured contracts of stock index futures can be predicted.
\item Statistical arbitrage opportunities can be detected by our models.
\item Criteria for entering transactions with arbitrage opportunities can be determined.
\item R packages with a complete set of documents will be created.
\item Technical report and presentation summarizing the work.
    \end{enumerate}
\end{frame}

\begin{frame}
    \frametitle{4. Deliverables}
{\bf{From sponsor to team:}}\\
\vspace{3pt}
    \begin{enumerate}
\item A list of portfolio of interest is needed
\item Sponsor`s historic transaction data is needed for modeling, testing, and prediction
\item Computing resources
\item We also expect weekly conference calls for inquiries and consulting

    \end{enumerate}
\end{frame}

\section{Problem Statement}

\begin{frame}
    \frametitle{Problem Statement}
{\bf{Data Analysis}} \\To discover arbitrage opportunities, its crucial to extract information from data of historic transactions and featured stock index prices. \\
We are trying to work out the hidden connection between past data and future trends and make predictions based on this.\\
\vspace{7pt}
{\bf{Financial Analysis}}\\ Based on intensive financial analysis, we seek for benchmarks of arbitrage opportunities for our sponsor.\\
\end{frame}

\section{Approach}
\begin{frame}
    \frametitle{1. Data Collection}
\begin{enumerate}
\item Our targeting data should be historic closing prices of contracts of Chinese stock index futures.
\item We choose {\bf{CSI300}} to be used in our study. CSI 300 is a weighted stock market index designed to replicate the performance of 300 stocks traded in China stock exchanges.
\end{enumerate}    
\end{frame}

\begin{frame}
    \frametitle{2. Mathematical Models:}
{\bf{Test for Stationarity:}}\\
\vspace{7pt}
1. Process data before testing\\
\vspace{5pt}
2. Use unit root test to check the property of stationarity\\
\vspace{10pt}
 {\bf{Remark:}} 
\\
\noindent a. A stationary process is a stochastic process whose joint probability distribution does not change when shifted in time or space.\\
b. Providing stationarity of our data, we can apply time series models to analyze data.\\
b. In statistics, a unit root test tests whether a time series variable is non-stationary using an autoregressive model.

\end{frame}


\begin{frame}
    \frametitle{2. Mathematical Models:}
{\bf{Time Series Analysis:}}\\
\vspace{7pt}
\begin{enumerate}
\item Simulate price spreads with AR (1) process model $$X_t=c+\phi X_t-1+\epsilon _t$$
\\
\item Apply GARCH (1,1) model to simulate the residual series of above$$\epsilon_t=h_tz_t, ~~~
h_t^2=\omega+\alpha y_{t-1}^2+\beta\sigma_{t-1}^2$$
Here, $z_t$ and $\epsilon_t$ are white noise sequences, and other coefficients satisfy basic assumptions of AR(1) and GARCH(1,1) models.
\item Predict future price spreads by plugging real present data into the model.
\end{enumerate}
\end{frame}
\begin{frame}
    \frametitle{3. Discover Arbitrage Opportunity}
{\bf{Basic ideas:}}\\
\begin{enumerate}
\item Conduct financial analysis to calculate cost and profit
\item Predict cash flows of investment
\item Set up criteria for entering and leaving transactions
\end{enumerate}
\end{frame}


\begin{frame}
    \frametitle{4. Test and Modification}
{\bf{Basic ideas:}}\\
\begin{enumerate}
\item Real market data will be plugged into the model in order to test the model.
\item Make modification and improvement to our model.
\end{enumerate}
\end{frame}

\section{Project Progress}
\begin{frame}
    \frametitle{Progress Report}
{\bf{Accomplishments:}}\\
\begin{enumerate}
\item Massive data has been collected.
\item Stationarity of data has been verified.
\item AR(1) and GARCH(1,1) models have been applied to our data.
\end{enumerate}
\end{frame}

\begin{frame}
    \frametitle{R code}
{\bf{Unit Root Test:}}\\
library(fUnitRoots)\\
unitrootTest(cpdiff)\\
lm.reg=lm(cp1110,cp1109)\\
summary(lm.reg)\\
lm.reg.resid=lm.reg\$resid\\
unitrootTest(lm.reg.resid)\\
\end{frame}
\begin{frame}
    \frametitle{R code}
{\bf{Time Series Models:}}\\
library(rugarch)\\
e=0\\
pred.series=array()\\
pred.vol=array()\\
pred.series=r[1:300]\\
vol=array()\\
for (m in 301:len)\{\\
  e=e+1\\
  garch11=ugarchspec(variance.model = list(model = ``fGARCH", garchOrder = c(1, 1),\\ submodel=``GARCH"), mean.model = list(armaOrder = c(1, 0), include.mean = FALSE),\\ distribution.model = ``norm")\\
  myfit=ugarchfit(garch11, r[e:(m-1)], out.sample = 0)\\
  forc=ugarchforecast(myfit, n.ahead=1)\\
  pred.series[m]=as.data.frame(forc,which=``sigma")[,2]\\
  pred.vol[m]=as.data.frame(forc, which=``sigma")[,1]\}

\end{frame}

\section{Conclusion}
\begin{frame}
    \frametitle{Conclusion}
{\bf{Outcomes so far:}}\\
\begin{enumerate}
\item The data of CSI300 collected satisfies stationarity.
\item AR(1) and GARCH(1,1) models are applicable to our data. 
\end{enumerate}
{\bf{Work remaining:}}\\
\begin{enumerate}
\item Price spreads prediction
\item Financial analysis with respect to sponsor`s portfolio
\item Model test
\end{enumerate}  
\end{frame}

\begin{frame}
    \frametitle{Conclusion}
{\bf{Recommendation for future research:}}\\
\begin{enumerate}
\item It's crucial to select accessible and applicable data carefully.
\item Before setting up mathematical models, one should study theoretical models in depth.\\
\end{enumerate}
\end{frame}
\begin{frame}
\begin{center}
\vspace{15pt}
\Huge{Thank you!}
\end{center}
\end{frame}
\end{document}
